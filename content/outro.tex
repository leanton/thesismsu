\nonumchapter{Выводы}

Основные результаты курсовой работы могут быть сформулированы следующим образом:
\begin{enumerate}
\item Разработана зеркально-симметричная структура, ближнепольный оптический отклик которой чувствителен к направлению круговой поляризации. Структура представляет собой ансамбль из трех равноудаленных эллиптических отверстий, проделанных в оптически непрозрачной золотой пленке.

\item Численным методом FDTD проведена оптимизация геометрических параметров разработанной структуры. В ходе оптимизации была выбрана структура с толщиной пленки 150 нм, расстоянием от отверстий до центра 300 нм, большой и малой полуосями 150 нм и 50 нм соответственно.

\item Изготовлена серия тонких пленок золота методом термического напыления и исследована толщина полученных пленок с помощью атомно-силового микроскопа для последующего изготовления плазмонных наноантенн.
\end{enumerate}