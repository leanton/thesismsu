\nonumchapter{Введение}

Дипломная работа посвящена изучению эффектов взаимодействия $ \pi $-димеров золотых наностержней, на основе которых была построена плазмонная линейка. В работе представлены аналитические и численные расчеты взаимодействия двух наностержней, а также экспериментальные результаты исследования спектров пропускания ансамблей из димеров золотых наностержней как функции расстояния между ними.
%Современные исследования в биосенсорике привели к развитию целого ряда задач и приложений, связанных с улучшением чувствительности оптических сенсоров на основе металлических наночастиц.
%Оптические свойства различных наноструктур нашли широкое применение для детектирования биологических объектов: от определения точного положения нуклеотидов в молекулах ДНК до измерения размеров молекул ДНК.  Так, с помощью резонанса локальных плазмон-поляритонов можно добиться очень большой чувствительности в таких задачах.

Измерение расстояний на наномасштабах является актуальной на сегодняшний день задачей. Доступные на сегодняшний день методы микроскопии (электронно-лучевая, атомно-силовая и другие) требуют специальной подготовки образцов, которая в случае биологических объектов приводит к их гибели. Помимо методов микроскопии, на данный момент существует две методики косвенного определения расстояния, которые могут быть использованы в биологических системах. Первый метод носит название ферстеровского переноса энергии. В данном методе используется перенос энергии между двумя хромофорами --- донором и акцептором. Безызлучательный перенос энергии происходит от донора, находящегося в возбужденном состоянии, на акцептор через ближнепольное диполь-дипольное взаимодействие. Эффективность этого процесса зависит от расстояния между объектами, что позволяет измерять расстояние как между двумя молекулами, так и между метками в одной макромолекуле. Расстояние, которое можно измерить с помощью такого метода, составляет 2 -- 6 нм. Второй метод связан с использованием резонансных особенностей локальных плазмон-поляритонов (ЛПП) димеров металлических наночастиц. Оказывается, что электромагнитное взаимодействие между наночастицами приводит к сдвигу спектрального положения резонанса ЛПП по отношению к положению резонанса ЛПП изолированных наночастиц такой же формы и размера. Величина сдвига однозначно зависит от энергии взаимодействия между наночастицами, которая, в свою очередь, зависит от расстояния между наночастицами. На основе этого свойства была разработана так называемая плазмонная линейка для измерения расстояний на наномасштабах, в том числе в биологических системах \cite{DNA}. Ранее были изучены плазмонные линейки на основе наночастиц различной формы: наносфер \cite{nanospheres}, нанодисков \cite{plasonrulereq}, нанопризм \cite{nanoprism}, наностержней \cite{nanorods} и других. Однако в предыдущих экспериментальных работах, во-первых, не уделялось внимание изучению плазмонных линеек на основе $ \pi $-димеров наностержней, и, во-вторых, роли возбуждения ЛПП высших порядков в $ \pi $-димерах.

Целями данной дипломной работы являются численное моделирование смещения резонанса ЛПП в $ \pi $-димерах золотых наностержней различной длины, экспериментальная характеризация смещения резонанса ЛПП в димерах золотых наностержней, изготовленных методом электронно-лучевой литографии, и сравнение теоретических и экспериментальных данных. В работе оценена возможность использования $ \pi $-димеров наностержней в качестве плазмонной линейки и определен диапазон расстояний, в котором плазмонная линейка может функционировать. Также в $ \pi $-димерах наностержней были выявлены резонансы ЛПП различных порядков и было проведено сравнение их поведения в зависимости от размеров наностержней и расстояния между ними в $ \pi $-димерах.