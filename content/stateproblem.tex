\section{Постановка задачи}

В результате обзора литературных источников можно сделать вывод о том, что изучение эффектов взаимодействия димеров различной геометрической формы  является актуальной на сегодняшний день задачей. При этом ранее систематически не исследовались резонансные особенности $ \pi $-димеров золотых наностержней, а также не исследовалось возбуждение резонансов высоких порядков в димерах наностержней.

В данной работе поставлены и решены следующие задачи:
\begin{enumerate}
\item Численное исследование смещения резонанса ЛПП в димерах золотых наностержней различной длины и вывод уравнения плазмонной линейки для $ \pi $-димера золотых наностержней.
\item Экспериментальные обнаружение и характеризация смещения резонанса ЛПП в $ \pi $-димерах золотых наностержней, изготовленных методом ЭЛЛ, и сравнение теоретических и экспериментальных данных.
\item Оценка возможности использования димеров наностержней в качестве плазмонной линейки и определение диапазона расстояний, в котором плазмонная линейка может функционировать. Сравнение характеристик $ \pi $-димеров как плазмонных линеек при их работе на ЛПП первого и второго порядка.
\end{enumerate}