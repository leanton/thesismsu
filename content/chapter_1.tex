\nonumchapter{Обзор литературы}
\section{Локальные поверхностные плазмон-поляритоны}
Поверхностными поляритонами называются волны, распространяющиеся вдоль границы раздела двух различных сред и существующие одновременно в них обеих. Поля, переносимые этими волнами, локализованы вблизи поверхности и затухают по обе стороны от нее \cite{Libenson}. Поверхностными плазмон-поляритонами (ПП) являются поверхностные волны, распространяющиеся вдоль границы раздела сред металл-диэлектрик.

Помимо бегущих вдоль поверхности раздела сред металл-диэлектрик ПП, локализованные поверхностные электромагнитные возбуждения могут существовать и в других объектах, таких как металлические частицы. Такие поверхностные возбуждения в ограниченной геометрии называются локальными поверхностными плазмонами (ЛПП). Частота ЛПП может быть определена в электростатическом приближении из решения уравнения Лапласа с подходящими граничными условиями. Применение электростатического приближения, не учитывающего эффектов запаздывания, возможно при условии, что характерный размер системы $ a $ много меньше длины волны $ \lambda $ возбуждающего излучения, $ a \ll \lambda $. Для металлических сферических частиц, в которых диэлектрическая проницаемость металла определяется формулой Друде:

\begin{equation}
\varepsilon (\omega) = 1 - \frac{\omega ^2 _p}{\omega ^2} 
\label{eq:EpsilonFreeElectron},
\end{equation}
частота ЛПП находится по формуле:
\begin{equation}
\omega _l = \omega _p \left(\frac{l}{\varepsilon _0 (l + 1) + l}\right)^{1/2},  
\label{eq:FrequencyLocalSPP}
\end{equation}
где $ \omega _p $ -- плазменная частота металла, $  l = 1, 2, 3... $ -- порядок сферических функций, $ \varepsilon _0 $ -- диэлектрическая проницаемость окружающей среды \cite{Zayats}. Для сфер радиуса $ a $ и объема $ V $ Лоренц вывел соотношение для электрической поляризуемости:

\begin{equation}
\alpha = \frac{3 (m^2 - 1)}{4 \pi (m^2 + 2)} V = \frac{m^2 - 1}{m^2 + 2} a^3,
\label{eq:polarizability}
\end{equation}
где $ m = n - ik $ -- комплексный показатель преломления, состоящий из показателя преломления $ n $ и коэффициента экстинкции $ k $.

Для эллипсоидных частиц с полуосями $ a $, $ b $ и $ c $ выполняется соотношение, которое дает три главных значения $ \alpha_1 $, $ \alpha_2 $ и $ \alpha_3 $ тензора поляризуемости:
\begin{equation}
\alpha _i = \frac{V}{4 \pi (L_i + \frac{1}{m^2 - 1})},
\label{eq:polarizabilityEllip}
\end{equation}
где $ L_i $ -- фактор деполяризации, зависящий от размеров осей эллипсоида. При произвольном выборе полуосей $ a $, $ b $ и $ c $ получим:
\begin{equation}
L_1 = \int_0^\infty \frac{a b c \mathrm{d} s}{2 (s + a^2)^{3/2} (s + b^2)^{1/2} (s + c^2)^{1/2}},
\label{eq:Lfactor}
\end{equation}
и при циклической перестановке ту же формулу для $ L_2 $ и $ L_3 $ \cite{LPP_Hulst}.

\section{Эффекты  <<плазмонной линейки>> при взаимодействии димера плазмонных наночастиц}

Современные исследования в области плазмоники сосредоточены в том числе на изучении резонанса ЛПП в ансамблях металлических наночастиц в связи с использованием их, например, в биосенсорике \cite{biosensing}. В этих исследованиях было показано, что электромагнитное взаимодействие между частицами приводит к сдвигу спектрального положения резонанса ЛПП по отношению к положению резонанса ЛПП изолированных наночастиц такой же геометрической формы.
Величина сдвига зависит от энергии взаимодействия между наночастицами, которая, в свою очередь, зависит от расстояния между наночастицами. Следовательно, измеряя величину сдвига резонанса, можно определить расстояние между наночастицами. Sonnichsen et al. \cite{bioplasmonruler} и Reinhard et al. \cite{bioplasmonruler2} использовали данное свойство для разработки так называемой  <<плазмонной линейки>> для измерения расстояний на наномасштабах, в том числе в биологических системах.  В статье \cite{plasonrulereq} исследуется зависимость локального плазмонного резонанса от расстояния между двумя нанодисками из золота. Пары нанодисков из золота располагались на кварцевой подложке и имели толщину 25 нм и диаметр 88 нм. Расстояние между нанодисками варьировалось и составляло 212, 27, 17, 12, 7 и 2 нм. Изображение ансамбля из спаренных нанодисков с расстоянием между нанодисками 12 нм, полученное с помощью растрового электронного микроскопа, показано на рис. \ref{img:PR_SEM}.
\begin{figure}
\center{\includegraphics[width=10cm]{img/PR_SEM.png}}
\caption{Изображение пар нанодисков из золота, полученное с помощью растрового электронного микроскопа. Расстояние между нанодисками составляет 12 нм, диаметр нанодиска составляет 88 нм, а толщина --- 25 нм \cite{plasonrulereq}.}
\label{img:PR_SEM}
\end{figure}
Спектры резонансов ЛПП были получены с помощью микроспектроскопии пропускания. %При этом были выбраны два направления поляризации падающего света: параллельное оси между центрами нанодисков и перпендикулярной ей.
При падении света с поляризацией, параллельной оси между центрами нанодисков, наблюдался сдвиг резонанса ЛПП в красную область спектра при уменьшении расстояния между нанодисками. На рис.~\ref{img:PR_extinction} показана зависимость коэффициента экстинкции от длины волны падающего света с поляризацией, параллельной оси между центрами нанодисков, для различных расстояний между нанодисками.
\begin{figure}[t]
\center{\includegraphics[width=18cm]{img/PR_extinction_ru.pdf}}
\caption{(а) Экспериментальные и (б) численные спектры коэффициента экстинкции при падении света на структуру из спаренных нанодисков из золота с различными расстояниями между нанодисками. Поляризация параллельна оси между центрами нанодисков \cite{plasonrulereq}.}
\label{img:PR_extinction}
\end{figure}
\begin{figure}[t]
\center{\includegraphics[width=17cm]{img/PR_ruler_ru.pdf}}
\caption{Зависимость сдвига резонанса ЛПП для пар золотых нанодисков от расстояния между нанодисками в эксперименте (а) и численных расчетах (б). Красная линия соответствует кривой $ y = y_0 + a \cdot e^{-x/l} $  \cite{plasonrulereq}.}
\label{img:PR_ruler}
\end{figure}

На рис.~\ref{img:PR_ruler} показана зависимость сдвига резонанса ЛПП от расстояния между нанодисками. Резонанс ЛПП при расстоянии между нанодисками 212 нм был выбран опорным при расчете относительного сдвига, так как взаимодействие между нанодисками можно положить минимальным.

Учитывая данное поведение системы, было выведено следующее феноменологическое уравнение для <<плазмонной линейки>>:
\begin{equation}
\frac{\Delta \lambda}{\lambda_0} \approx 0.18 \cdot \exp \left( \frac{-(s/D)}{0.23} \right),
\label{eq:plasmon_ruler}
\end{equation}
где $ \Delta \lambda $ -- величина сдвига резонанса ЛПП, $ \lambda_0 $ -- резонанс ЛПП изолированной наночастицы, $ D $ -- диаметр наночастицы. Такая линейка, например, была использована для измерения длин молекул ДНК \cite{bioplasmonruler3}.

В других работах также исследовалась зависимость сдвига резонанса ЛПП в зависимости от расстояния между наночастицами. При этом частицы были различной геометрической формы, например, нанопризмы \cite{nanoprism, nanoshells}, нанокольца \cite{nanoring}, наносферы \cite{nanospheres, nanospheres2}, наноциллиндры \cite{nanocyllinders} и наностержни \cite{nanorods}. Также исследовался резонанс ЛПП при различных взаимных расположениях двух наночастиц одной геометрической формы \cite{nanorods2, nanorods3, 3druler}. При этом падающий свет был поляризован параллельно оси между центрами наночастиц. Однако обычно изучалась только одна мода, поэтому представляет большой интерес изучение поведения резонанса ЛПП для частиц другой геометрии или взаимного положения, а также взаимодействия наночастиц, в которых возможно возбуждение различных плазмонных мод.