\section{Обсуждение результатов}

Рабочим диапазоном плазмонной линейки в данной работе является расстояние между наностержнями от $ d = $ 0 нм до такого значения резонанса ЛПП $ \lambda_0 (d)$, при котором производная $ \lambda'_0 (d) $  в первый раз равняется нулю. Из численных результатов (рис.~\ref{img:res_comparison}) было получено, что верхняя граница применимости плазмонной линейки в случае резонанса первого порядка составляет $ \approx 170 \pm 10 $ нм, а в случае резонанса второго порядка она больше и составляет $ \approx 370 \pm 10 $ нм. Из экспериментальных данных мы видим, что для резонанса ЛПП второго порядка экспериментальная кривая положения резонанса ЛПП от расстояния между наностержнями согласуется с численными расчетами за исключением того, что в абсолютных значениях экспериментально полученные данные положения резонанса ЛПП сдвинуты в красную область на  $ \approx 25 $ нм в случае длины наностержней $ a = $ 400 нм и на $ \approx 50 $ нм в случае длины наностержней $ a = $ 500 нм. Это связано с тем, что численном расчете не учитывалась кварцевая подложка, которая влияет на положение резонанса ЛПП. В случае резонанса первого порядка ЛПП экспериментальные данные не согласуются с численными расчетами (рис.~\ref{img:1res}). Это связано с дифракцией излучения на периодически расположенных $ \pi $-димерах в исследуемых образцах. В случае же численного эксперимента при расчете зависимости положения резонанса ЛПП от расстояния между наностержнями использовались граничные условия PML, которые с физической точки зрения поглощают все падающее на них электромагнитное поле. Для того, чтобы учесть периодичность структуры в были проведены численные расчеты спектров коэффициента экстинкции для $ \pi $-димеров с фиксированной длиной наностержней $ a = $ 300 нм и расстоянием между наностержней $ d = $ 300 нм от расстояния между $ \pi $-димерами (рис.~\ref{img:BCperiod}). Для этого граничные условия PML были изменены на периодические граничные условия. Видно, что резонанс второго порядка менее чувствителен к периоду между $ \pi $-димерами, в то время как положение резонанса первого порядка начинает сильно варьироваться из-за дифракционного взаимодействия между $ \pi $-димерами, а иногда происходит расщепление резонанса ЛПП. Аналогичный эффект дифракционного взаимодействия наблюдался в случае периодически расположенных наноантенн в статье \cite{diffractionCoupling}. Наноантенны представляли из себя нанокольца и наблюдалось два резонанса: дипольный и квадрупольный. В итоге квадрупольный резонанс оказался менее чувствителен к дифракционному взаимодействия наноантенн.
Таким образом, дифракционное взаимодействие $ \pi $-димеров и дальнепольное взаимодействие между наностержнями в димере приводят к тому, что верхняя граница применимости $ \pi $-димера в случае резонанса первого порядка меньше, чем для резонанса второго порядка. 