\nonumchapter{Выводы}

Основные выводы могут быть сформулированы следующим образом:
\begin{enumerate}
\item Построена аналитическая модель смещения резонанса локальных плазмон-поляритонов в димерах золотых наностержней.  В качестве модели использовалось взаимодействие двух точечных диполей, диэлектрическая проницаемость которых описывается формулой Друде. Была получена аналитическая формула зависимости резонанса локальных плазмон-поляритонов от расстояния между наностержнями, а также показано существование двух мод: симметричной и антисимметричной.
\item Численно исследовано методом конечных разностей во временной области  смещение резонанса локальных плазмон-поляритонов в димерах золотых наностержней. В ходе исследования было обнаружено наличие двух резонансов локальных плазмон-поляритонов с различным локальным распределением плотности мощности электромагнитного поля для наностержней длиной до 500 нм. На основе смещения положения резонансов первого и второго порядков были получены уравнения для <<плазмонных линеек>>.
\item Получена верхняя граница возможности использования рассматриваемых димеров золотых наностержней в качестве <<плазмонной линейки>>. Для резонанса первого порядка она составляет $  150 \pm 10 $ нм, а для резонанса второго порядка --- $ 400 \pm 30 $ нм.
\item Методами микроспектроскопии пропускания экспериментально охарактеризованы смещения резонанса локальных плазмон-поляритонов.  При этом для длин наностержней 100, 150 и 200 нм наблюдался резонанс первого порядка, а для наностержней длиной 400 и 500 нм --- резонанс второго порядка. Экспериментальные данные смещения резонанса второго порядка сходятся в пределах погрешности с результатами численного расчета, что говорит о возможности использования данной системы в качестве <<плазмонной линейки>>.

\end{enumerate}