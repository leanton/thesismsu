\nonumchapter{Выводы}

Основные выводы работы могут быть сформулированы следующим образом:
\begin{enumerate}
\item Методом конечных разностей во временной области исследовано смещение резонанса локальных плазмон-поляритонов в $ \pi $-димерах золотых наностержней. Обнаружено два резонанса локальных плазмон-поляритонов --- первого и второго порядков --- с различным локальным распределением плотности мощности электромагнитного поля вблизи димеров для наностержней длиной от 100 нм до 500 нм. На основе смещения положения резонансов первого и второго порядков получены уравнения для плазмонных линеек. Численно определена верхняя граница применимости рассматриваемых димеров золотых наностержней в качестве плазмонной линейки, составившая $ 145 \pm 25 $~нм для резонанса первого порядка и $ 380 \pm 30 $ нм для резонанса второго порядка.
\item Методами микроспектроскопии пропускания экспериментально охарактеризовано\\ смещение резонансов локальных плазмон-поляритонов в образцах $ \pi $-димеров золотых наностержней.  Для длин наностержней 100, 150, 200 и 300 нм наблюдался резонанс первого порядка, а для наностержней длиной 400 и 500 нм --- резонанс второго порядка. Экспериментальные данные смещения резонанса второго порядка сходятся в пределах погрешности с результатами численного расчета, в то время как резонанса первого порядка совпадения экспериментальных и численных результатов не наблюдается.
\item Проведено сравнение полуаналитических расчетов, численных расчетов и экспериментальных данных. Из численных и полуаналитических расчетов установлено, что вследствие дальнепольного взаимодействия верхняя граница применимости $ \pi $-димера золотых наностержней в качестве плазмонной линейки меньше для резонанса первого порядка, чем для резонанса второго порядка. Это связано с большей чувствительностью резонанса локальных плазмон-поляритонов первого порядка к дальнепольному взаимодействию. Из численных расчетов с периодическими граничными условиями и экспериментальных данных следует, что из-за дифракционного взаимодействия периодически расположенныъ $ \pi $-димеров использование резонанса первого порядка для построения плазмонной линейки невозможно. Резонанс локальных плазмон-поляритонов второго порядка оказался менее чувствителен к дифракционному взаимодействию $ \pi $-димеров, поэтому его использование для построения плазмонной линейки более целесообразно.

\end{enumerate}

Автор выражает особую благодарность научному руководителю Максиму Радиковичу Щербакову за постановку актуальной и интересной задачи, поддержку с его стороны в проведении экспериментов и расчетов, Артему Вячеславовичу Четвертухину за помощь в получении численных результатов, а также всем студентам, аспирантам и сотрудникам лаборатории нанооптики метаматериалов, принимавшим активное участие в обсуждении результатов.