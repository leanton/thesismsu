\nonumchapter{Выводы}

Основные выводы могут быть сформулированы следующим образом:
\begin{enumerate}
%\item Построена аналитическая модель смещения резонанса локальных плазмон-поляритонов в димерах золотых наностержней.  В качестве модели использовалось взаимодействие двух точечных диполей, диэлектрическая проницаемость которых описывается формулой Друде. Была получена аналитическая формула зависимости резонанса локальных плазмон-поляритонов от расстояния между наностержнями, а также показано существование двух мод: симметричной и антисимметричной.
\item Методом конечных разностей во временной области исследовано смещение резонанса локальных плазмон-поляритонов в $ \pi $-димерах золотых наностержней. В ходе исследования было обнаружено наличие двух резонансов локальных плазмон-поляритонов с различным локальным распределением плотности мощности электромагнитного поля вблизи димеров для наностержней длиной от 100 нм до 500 нм. На основе смещения положения резонансов первого и второго порядков были получены уравнения для плазмонных линеек. Численно получена верхняя граница применимости рассматриваемых димеров золотых наностержней в качестве плазмонной линейки. Для резонанса первого порядка она составляет $  170 \pm 10 $ нм, а для резонанса второго порядка --- $ 370 \pm 10 $ нм.
\item Методами микроспектроскопии пропускания экспериментально охарактеризованы смещения резонанса локальных плазмон-поляритонов.  При этом для длин наностержней 100, 150, 200 и 300 нм наблюдался резонанс первого порядка, а для наностержней длиной 400 и 500 нм --- резонанс второго порядка. Экспериментальные данные смещения резонанса второго порядка сходятся в пределах погрешности с результатами численного расчета, что говорит о возможности использования данной системы в качестве плазмонной линейки. 
\item Проведено сравнение аналитических расчетов, численных расчетов и экспериментальных данных. При этом было выявлено, что в случае резонанса первого порядка дифракционное взаимодействие $ \pi $-димеров делает возможным использование $ \pi $-димера золотых наностержней в качестве плазмонной линейки только на расстояних до 100 нм между наностержнями. Резонанс локальных плазмон-поляритонов оказался менее чувствительным к дифракционному взаимодействию $ \pi $-димеров, поэтому его использование как плазмонной линейки более целесообразно.

\end{enumerate}